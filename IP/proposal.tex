\documentclass[12pt,a4paper]{article}
% The following LaTeX packages must be installed on your machine: amsmath, authblk, bm, booktabs, caption, dcolumn, fancyhdr, geometry, graphicx, hyperref, latexsym, natbib
\input{185.dat}
\usepackage{gensymb}
\usepackage{amsthm}
\usepackage{float}
\usepackage{siunitx}
\usepackage{amssymb}
\usepackage{float}
\usepackage{enumerate}
\usepackage{listings}
\usepackage{mathtools}
\PassOptionsToPackage{hyphens}{url}\usepackage{hyperref}
\usepackage[none]{hyphenat}
\usepackage{physics}
\newcommand\ddfrac[2]{\frac{\displaystyle #1}{\displaystyle #2}}
%\renewcommand{\familydefault}{\sfdefault}


\begin{document}

\begin{titlepage}
\begin{center}
\vspace*{\fill}

\Huge{ IP Proposal \\
29 April 2019
} \\

\qquad
\qquad

\normalsize{Kenneth V. Domingo \\
Rhei Joven G. Juan \\
Rene L. Principe Jr. \\ \bigskip
App Physics 185 (A.Y. 2018-19)
}

\vspace*{\fill}
\end{center}
\end{titlepage}

\setcounter{page}{1}

\section{Objectives}
\begin{itemize}
	\item To be able to detect the force exerted by various locations on a person's foot.
	\item To be able to characterize, based on the force vs time graph, whether a person is standing, walking, running, or jumping.
	\item To be able to characterize the sport being played based on the force vs time graph.
\end{itemize}

\section{Materials \& Methodology}

\begin{table}[h!]
	\centering
	\caption{Estimated cost of materials.}
	\begin{tabular}{cccc}
	Quantity & Item & Cost/pc (PhP) & Subtotal (PhP) \\ \hline
	1 & Pressure-sensitive conductive velostat sheet & 349.00 & 349.00 \\
	1 & ESP8266 WiFi Microcontroller & 325.00 & 325.00 \\
	2 & 9V Battery & 79.00 & 158.00 \\
	& & \textbf{TOTAL} & 832.00
	\end{tabular}
\end{table}

\begin{enumerate}
	\item Pieces of the velostat sheet are attached to 3 locations on the insole of a shoe. 
	\item The velostat is attached to the microcontroller for real-time data acquisition.
	\item Voltage vs. force calibration was done by applying different values of force to the velostat sheet.
	\item A force vs. time is plotted in real-time for different motions.
	\item A characterization of the different motions was done using the frequencies and magnitudes obtained from the force vs. time plot.
\end{enumerate}

\section{Predicted Results}
Characterization of different motions:
\begin{itemize}
	\item \textbf{Standing} - constant force observed over time
	\item \textbf{Walking} - a cascading motion from the three sensors will be observed
	\item \textbf{Running} - same as walking but with higher frequency
	\item \textbf{Jumping} - two sets of impulse will be observed from the take-off and landing.
\end{itemize}
\section{Validation Scheme}
\begin{itemize}
	\item Compare the obtained force value from the sensor with the calculated calibration equation/curve, for standing motion.
	\item Through real-time analysis, compare a video of the motions to their force vs. time plot.
\end{itemize}

\end{document}